\documentclass[10pt, oneside]{jarticle}   	% use "amsart" instead of "article" for AMSLaTeX format
\usepackage{geometry}                		% See geometry.pdf to learn the layout options. There are lots.
\geometry{a4paper}                   		% ... or a4paper or a5paper or ... 
%\geometry{landscape}                		% Activate for for rotated page geometry
%\usepackage[parfill]{parskip}    		% Activate to begin paragraphs with an empty line rather than an indent
%\usepackage{graphicx}				% Use pdf, png, jpg, or eps§ with pdflatex; use eps in DVI mode
								% TeX will automatically convert eps --> pdf in pdflatex		

\usepackage[all]{xy}
\usepackage{enumerate}
\usepackage{version}
\usepackage{amssymb}
\usepackage{amsmath}
\usepackage{cases}

\usepackage[dvipdfmx]{graphicx}


\usepackage{amssymb}
\usepackage{amsthm}

\include{"texdefinition"}

\title{$\forall$の場合の反証について}
\author{H2nI3sc}
%\date{}							% Activate to display a given date or no date

\begin{document}
\maketitle

\section{何を書きたいか}
\subsection{意図}
conjectureが$\exists x P(x)$の場合は、$\mathcal{A} \cup P(x)$から$\cont$を1つ証明すればよい。
しかし、conjectureが$\forall x P(x)$の場合は、必ずしもそうとはいえない。
そのことについて書く。

\include{"definition"}

\newpage

\section{反証では何をする必要があるのか}
観測集合$\mathcal{A}$のconjectureとして $\forall x P(x)$を証明するには
conjectureの否定から $\mathcal{A} \cup \eset{\exists x \bar{P}(x)}$から$\cont$を証明することになる。

この$\exists x$の$x$は、引数となる$forall $変数がないので、どこかから定数をとってくることになる。
その定数は、$\herbrand$でなければ、resolutionが適用できないので、$\mathcal{H}$から持ってくることになるだろう。

故に、反証の対象となるconjectureの否定は$\herbrand$の要素のそれぞれについて作る必要がある。
つまり、すべての $ c \in \herbrand$ について次の観測集合から$\cont$を証明しなくてはならない。

\begin{equation}
 \mathcal{A} \cup \{ \bar{P}(c) \} 
\end{equation}

もしも$\clos{A}$にconjecureのgeneralな式が含まれていれば、conjectureはそこから証明できるので、このような反証さがしは必要ない。
つまり

\begin{description}
\item $\mathcal{A}$にconjecture自身が書かれているか、さもなければ
\item $\herbrand$のすべての要素について反証を行う
\end{description}

ことになる。

\end{document}


