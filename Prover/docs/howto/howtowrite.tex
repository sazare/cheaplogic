\documentclass[10pt, oneside]{jarticle}   	% use "amsart" instead of "article" for AMSLaTeX format
\usepackage{geometry}                		% See geometry.pdf to learn the layout options. There are lots.
\geometry{a4paper}                   		% ... or a4paper or a5paper or ... 
%\geometry{landscape}                		% Activate for for rotated page geometry
%\usepackage[parfill]{parskip}    		% Activate to begin paragraphs with an empty line rather than an indent
%\usepackage{graphicx}				% Use pdf, png, jpg, or eps§ with pdflatex; use eps in DVI mode
								% TeX will automatically convert eps --> pdf in pdflatex		

\usepackage[all]{xy}
\usepackage{enumerate}
\usepackage{version}
\usepackage{amssymb}
\usepackage{amsmath}
\usepackage{cases}

\usepackage[dvipdfmx]{graphicx}


\usepackage{amssymb}
\usepackage{amsthm}

\theoremstyle{definition}
\newtheorem{theorem}{定理}
\newtheorem{proposition}{命題}
\newtheorem{hypothesis}{予想}
\newtheorem{problem}{課題}
\newtheorem*{theorem*}{定理}
\newtheorem*{proposition*}{命題}
\newtheorem*{hypothesis*}{予想}
\newtheorem{definition}[theorem]{定義}
\renewcommand\proofname{\bf 証明}


\newcommand{\ivec}[2]{\bar{#1}^{(#2)}}
\newcommand{\similarity}[1]{\rho(#1) }
\newcommand{\cor}[2]{<\bar{#1}\mid\bar{#2}>}

\newcommand{\almostall}{\bar{\forall}}
\newcommand{\possibly}{\bar{\exists}}

\newcommand{\undet}{\omega}
\newcommand{\cont}{\Box}

\newcommand{\eset}[1]{\{{#1}\}}
\newcommand{\clos}[1]{\mathcal{#1}^{*}}


\title{述語論理での世界の書き方の研究}
\author{H2nI3sc}
%\date{}							% Activate to display a given date or no date

\begin{document}
\maketitle

\section{何を書きたいか}
\subsection{意図}


\subsection{用語あるいは定義}
\begin{description}
\item[観測命題:] 正解を観測して得られた事実とみなせる論理式。ここでは観測と呼んでいるが、数理論理学では公理と呼ばれているもの。(ここ、意味論と証明論がまざっている)公理というほど確定したものではなく、観測し正しいと信じられる言明程度のもの。
\item[観測集合:] 観測命題の集合。記号では$\mathcal{A}$で表すと思う。
\item[仮説:] 証明しようとする論理式。conjectureと呼んだり、conjと書いたりする。記号では$\Psi(x)$のように書くことが多いはず。
\item[Fact:] 観測論理式の中で、観測された事実のみの論理式やその集合を意味する。真実という意味ではない。ground unit clause になる。
\item[定義:] 観測集合に書かれた、Factの間の関係を意味する論理式。indubtive definition も含むが、定義という役割にすることで、真偽を証明する義務を除外したものが定義。
\item[エルブラン宇宙] エルブランドメインと呼ぶかも。特定の述語記号に対するエルブランドメインについても書くかもしれない。
\end{description}

\subsection{定義}

$\cont$は 矛盾を示す。

$\mathcal{H_{\mathcal{A}}}$は、観測集合$\mathcal{A}$に対するエルブラン宇宙

$\clos{A} := \{x | \mathcal{A} \vdash x\}$



\section{有限の場合}
\subsection{基本的な検討}
Factとは、事実としての根拠がある命題の集合だと考える。
その根拠は論理の外側で判定される。たとえば、Neural Network(以下ではNN)などを想定。
NNでFact集合を構成する方法は未知。

$\mathcal{A}$にはFact以外の命題も含まれる。
命題レベルの推論である場合も、変数を含む場合もある。
NNでそのような命題が構成できるのかは未知。

\subsection{$\mathcal{A}=\eset{}$について}
観測集合が$\mathcal{A}=\{\}$の場合は、リテラルが1つもないので、どんなconjectureも観測命題とresolutionできない。
つまり、$\mathcal{A}$に含まれない述語記号や定数記号を用いて構成したどのような$\Psi$についても、$\{\} \nvdash \Psi$ である。
このとき、refutationを考えると、$\cont$を証明することが不可能なので、$\{\}$は無矛盾である。

(個人的な感想)無矛盾というと、直感的には$\Psi$か$\bar\Psi$かどちらかが証明できて、両方一緒に証明はできないという話だと思う。
この場合、$\Psi$も$\bar\Psi$も証明できないので、無矛盾というのとはすこし違うイメージがある。
数学での$\empty$にまつわる証明はどれもそういうものだけれど。

証明の作り方から、conjectureのリテラルもその否定も$\mathcal{A}$に出現していない場合、それを真とも偽ともみなせないということなので、真偽値に値も含めて考えてみる。

それを$\undet$と書くことにする。

つまり、真偽値を$\{\top, \bot, \undet\}$とする。

\subsection{$\mathcal{A}=\eset{P(a)}$}

これに対して、次のようなconjの場合の真偽値は次の通り。

\begin{table}[htbp]
 \centering
 \begin{tabular}{|c|c|c|c|}\hline
   No & 観測 & $\Psi$ & 真偽値 \\ \hline
   1 & $\eset{P(a)}$ & $P(a)$ & T \\ \hline
   2 & $\eset{P(a)}$ & $\bar P(a)$ &$\undet$  \\ \hline
   3 & $\eset{P(a)}$ & $P(b)$ & $\undet$ \\ \hline
   4 & $\eset{P(a)}$ & $\bar P(b)$ & $\undet$ \\ \hline
   5 & $\eset{P(a)}$ & $Q(a)$ & $\undet$ \\ \hline
   6 & $\eset{P(a)}$ & $\bar Q(a)$ & $\undet$ \\ \hline
   7 & $\eset{P(a)}$ & $\exists x P(x)$ & $T \, \because \exists a \, P(a)$ \\ \hline
   8 & $\eset{P(a)}$ & $\forall x P(x)$ & $T  \, \because P(a) $ a  is all of  H \\ \hline
 \end{tabular}
 \caption{集合と仮説の関係(有限の場合1)}
 \label{tab:ex1}
\end{table}

ただし、$H_{\mathcal{A}} = \eset{a}$


まず、未定義について考えると、次の3つの場合がある。
\begin{enumerate}
\item Pという述語記号が$\mathcal{A}$に出現しない。
\item conjectureに出現するリテラルのcomplementが$\mathcal{A}$に出現しない。この場合は、片方だけは真偽が決まるが、その反対は未定義になる。
\item conjectureに出現するリテラルの述語記号が$\mathcal{A}$に出現しない。
\end{enumerate}

2は、complementが存在しないだけで、$P$($P(a)$)は$\mathcal{A}$に出現している。
3,4は、述語記号は$\mathcal{A}$に出現しているが、resolution によって消滅させるリテラルが存在しないので、証明も反証も不可能であり、未定義とする。
5,6は、述語記号Q自体が$\mathcal{A}$に出現していない。

refutationで$\cont$が導けないときは、conjectureが導けない、つまりということであり、これを偽だとみなすすれば、$\undet$は$F$と同じ。

表では、区別するために$\undet$にしておく。

7については、エルブラン宇宙のterm aについて$P(a)$が成り立つので$\{\exists x P(x)\}$ が真となる。

8は、意味論で考えると、エルブラン宇宙のすべてのterm、つまりただ一つのterm aについて$P(a)$が成り立つので$\{\forall x P(x)\}$は真となる。

証明の観点から見ると、resolution refutaionを用いる場合、conjectureを否定し、観測集合との和をとって$\cont$を導けるかどうかを試すので、7の場合は$\neg \forall x P(x)$から、$\bar P(x)$というclauseをつくり、これを$\mathcal{A}$に加えて

$\{P(a), P(x)\} \vdash \cont$ 

これが成り立つので、conjectureについては真となる。

8の場合は、conjectureを否定すると $\neg \{\forall x P(x)\}$ から $\exists x \bar P(x)$がconjの否定となり、この$\exists x$の$x$を、定数に置き換える。
その定数($\exists$の前に$\forall$があれば、関数だが、ないので0引数の関数として定数になる)はconjectureの論理式に出現していない定数である必要があるが、aはその条件を満たしている唯一のtermなので $\bar P(a)$が最終的にconjの否定となる。
(conjectureに含まれない定数という意味では、$\mathcal{H}$にない定数、たとえばここでは b でもよいが、その場合unifiableなリテラルが存在しないことは明らかなので、そのような定数を導入する必要はない。つまり、エルブラン宇宙の要素だけを考えれば良い)

これから

$\eset{P(a), P(a)} \vdash \cont$ 

という反証ができるので、conjectureは真となる。

(ここらへん意味論と証明論がごっちゃになっているかも)


\subsection{$\mathcal{A}=\eset{P(a), P(b)}$}
これに対して、次のようなconjの場合の真偽値は表の通り。

\begin{table}[htbp]
 \centering
 \begin{tabular}{|c|c|c|c|}\hline
   No & 観測 & $\Psi$ & 真偽値 \\ \hline
   1 & $\eset{P(a), P(b)}$ & $P(a)$ & T \\ \hline
   2 & $\eset{P(a), P(b)}$ & $\bar P(a)$ & $\undet$  \\ \hline
   3 & $\eset{P(a), P(b)}$ & $P(b)$ & T \\ \hline
   4 & $\eset{P(a), P(b)}$ & $\bar P(b)$ & $\undet$  \\ \hline
   5 & $\eset{P(a), P(b)}$ & $Q(a)$ & $\undet$ \\ \hline
   6 & $\eset{P(a), P(b)}$ & $\bar Q(a)$ & $\undet$ \\ \hline
   7 & $\eset{P(a), P(b)}$ & $\exists x P(x)$ & $T \because P(a)$ \\ \hline
   8 & $\eset{P(a), P(b)}$ & $\forall x P(x)$ & $T \because  P(a) and P(b) hold$ \\ \hline
 \end{tabular}
 \caption{集合と仮説の関係(有限の場合2)}
 \label{tab:ex2}
\end{table}

観測集合で$P(b)$が増えたので、3も真偽値はTになった。

しかし、$Q$は含まれないので、5,6は$\undet$のまま。

7については、conjの否定は同じ形になるので、証明すべきは

$\eset{P(a), P(b), P(x)} \vdash \cont$ 

であるが、これは成り立つ。

ただし、反証は2つ存在するが、$\exists$の意味から、2つ目の証明は考慮されない。この場合はb。

8については、conjの否定$\exists x \bar P(x)$の変数$x$は、$\mathcal{H}$からaとbの二つが存在する。

$\{P(a), P(b), \bar P(a)\} \vdash \cont$

$\{P(a), P(b), \bar P(b)\} \vdash \cont$

一般には、どちらかが証明できないかもしれないので、
証明はすべての場合について判定する必要がある。
この場合は、どちらであっても矛盾するので、片方だけでも正解になるが、一般にはそうはいかないという話。


つまり証明の対象とする集合のすべてについて反証が必要ということであり、$\forall$の意味が証明手続きに移行したということになる。

この議論は、$\eset{P(a), P(b), P(c)}$など定数が増えても同じ。
\subsection{$\mathcal{A}=\{P(a), Q(a)\}$}
この場合は、$P(a)$の場合と違わないので、省略する

\subsection{$\mathcal{A}=\eset{P(a), Q(b)}$}
これは、述語記号が増えた場合。

\begin{table}[htbp]
 \centering
 \begin{tabular}{|c|c|c|c|}\hline
   No & 観測 & $\Psi$ & 真偽値 \\ \hline
   1 & $\eset{P(a), Q(b)}$ & $P(a)$ & T \\ \hline
   2 & $\eset{P(a), Q(b)}$ & $\bar P(a)$ & F \\ \hline
   3 & $\eset{P(a), Q(b)}$ & $P(b)$ & $\undet$ \\ \hline
   4 & $\eset{P(a), Q(b)}$ & $\bar P(b)$ & $\undet$ \\ \hline
   5 & $\eset{P(a), Q(b)}$& $\exists x P(x)$ & $T \because P(a)$ \\ \hline
   6 & $\eset{P(a), Q(b)}$ & $\forall x P(x)$ & $F \because b \in H \land \bar P(b)$ \\ \hline
 \end{tabular}
 \caption{集合と仮説の関係(有限の場合2)}
 \label{tab:ex3}
\end{table}

ここではエルブラン宇宙に b が存在するが、P(b) というリテラルが観測集合にでていないので、未定義としている。

5は、

$\eset{P(a), Q(b), P(x)} \vdash \cont$ 

の証明を求めることになるが、この場合、反証は $\cont\{x \leftarrow a\}$のみとなる。

6は、

$\{P(a), Q(b), \bar P(a)\} \vdash \cont$

$\{P(a), Q(b), \bar P(b)\} \vdash \cont$

の両方の証明が必要だが、$P(b)$の反証に失敗するので、conjecture は否定される。


\subsection{$\mathcal{A}=\eset{P(a), \bar P(a)Q(b)}$}

これは、要素数が2の観測集合だが、命題の推論ができるので、$\clos{A}=\eset{P(a), \bar P(a)Q(b), Q(b)}$となる。

\begin{table}[htbp]
 \centering
 \begin{tabular}{|c|c|c|c|}\hline
   No & 観測 & $\Psi$ & 真偽値 \\ \hline
   1 & $\{P(a), \bar{P}(a)Q(b)\}$ & $P(a)$ & T \\ \hline
   2 & $\{P(a), \bar P(a)Q(b)\}$ & $\bar P(a)$ & $\undet$ \\ \hline %? F
   3 & $\{P(a), \bar P(a)Q(b)\}$ & $Q(b)$ & T \\ \hline
   4 & $\{P(a), \bar P(a)Q(b)\}$ & $\bar Q(b)$ & $\undet$ \\ \hline  % ? F
   5 & $\{P(a), \bar P(a)Q(b)\}$ & $\exists x Q(x)$ & $T \because Q(b) \in \clos{A}$ \\ \hline
   6 & $\{P(a), \bar P(a)Q(b)\}$ & $\forall x Q(x)$ & $F \because  \bar Q(a) \notin \clos{A}$ \\ \hline
   7 & $\{P(a), \bar P(a)Q(b)\}$ & $\exists x P(x)$ & $T \because P(a) \in \clos{A}$ \\ \hline
   8 & $\{P(a), \bar P(a)Q(b)\}$ & $\forall x P(x)$ & $F \because \bar P(b) \notin \clos{A}$ \\ \hline
 \end{tabular}
 \caption{集合と仮説の関係(有限の場合2)}
 \label{tab:ex4}
\end{table}

%%% ちょっと混乱してきた
%%% P(a)はあるが-P(a)がないとき、P(a)は undetになるのかFになるのか?? Fなのか???
%%% Pがまったくないとき +/-P(a)はundetになる

$\bar P(a) Q(b)$は、変数がないので、命題論理と同じ推論しかされないが、conjectureが変数を含む形の場合は、変数への代入を生むので、必ずしも命題論理に限定した話ではなくなる。

$\bar P(x) P(f(x))$という形の無限にconsequenceを生成するclauseについては次のセクションか付録で分析する。


\subsection{有限のまとめ}
conjecture が$\exists x $のprefixを持つ場合、resolution refutationによって、具体的な反例をみつけることができるので、通常のresolution refutationに基づく単純な証明方法で証明が可能である。

しかし、conjectureが$\forall x$のprefixを持つ場合、エルブラン宇宙の定数すべてについて、反証がないことを示さなくてはならないため、1つの観測集合の反証だけで終わらず、エルブラン宇宙のすべての要素をxに代入したconjectureのインスタンスについて、反証がないことを示さなくてはならない。
(もしも、$\mathcal{A}$にinductive definitionが書かれていたら、通常の証明手続きで反証できる)

1つの定数の反証は、必ず有限時間で終了するので、Factが有限集合の場合は実行可能である。

(心の声)

これは、エルブラン宇宙上での制御を必要とし、意味論のからんだ話になっているように見える。
ただし、エルブラン宇宙は機械的に作れるので、証明論の範疇なのかもしれない。

いずれにせよ、conjectureの形は同じでも、その意味は観測集合の形によって違ってくる。

モデルに依存した真偽値になっているのは、意味論の範疇なのか。

$\forall x$のprefixを持つconjectureの場合、反例がみつかると、それはこのconjectureのどこが違っているのかの情報になる。その情報が必要であれば、エルブラン宇宙のすべての要素について、矛盾のチェックが必要であり、そのような制御をすると$\forall x$と同じことになりそう。


\section{無限の場合}

$\clos{A}$が無限集合の場合について考える。

無限の観測集合$\mathcal{A}$は、有限時間で書くことはできないが、ここではそれを含めて検討する。

というのも、三つの場合を想定している。
\begin{enumerate}
\item $\mathcal{A}$が無限集合。仮想の状況を考えることで、思考実験として、無限の観測命題が書かれていると考える
\item $\mathcal{A}$は有限集合。観測命題の中に、induction definitionの形のものがあり、$\mathcal{A}$は有限だが$\clos{A}$が無限になる場合。
\item 
\end{enumerate}


\subsection{無限集合の場合}
\subsection{有限集合でルールが書かれている場合}

\section{付録: 計算過程}
\subsection{関数やプログラムの定義が書かれている場合}
\subsubsection{計算過程と関数定義}

\end{document}


