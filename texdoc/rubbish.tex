\documentclass[10pt, oneside]{jarticle}   	% use "amsart" instead of "article" for AMSLaTeX format
\usepackage{geometry}                		% See geometry.pdf to learn the layout options. There are lots.
\geometry{a4paper}                   		% ... or a4paper or a5paper or ... 
%\geometry{landscape}                		% Activate for for rotated page geometry
%\usepackage[parfill]{parskip}    		% Activate to begin paragraphs with an empty line rather than an indent
%\usepackage{graphicx}				% Use pdf, png, jpg, or eps§ with pdflatex; use eps in DVI mode
								% TeX will automatically convert eps --> pdf in pdflatex		

\usepackage[all]{xy}
\usepackage{enumerate}
\usepackage{version}
\usepackage{amssymb}
\usepackage{amsmath}
\usepackage{cases}

\usepackage[dvipdfmx]{graphicx}


\usepackage{amssymb}
\usepackage{amsthm}

\theoremstyle{definition}
\newtheorem{theorem}{定理}
\newtheorem{proposition}{命題}
\newtheorem{hypothesis}{予想}
\newtheorem{problem}{課題}
\newtheorem*{theorem*}{定理}
\newtheorem*{proposition*}{命題}
\newtheorem*{hypothesis*}{予想}
\newtheorem{definition}[theorem]{定義}
\renewcommand\proofname{\bf 証明}


\newcommand{\ivec}[2]{\bar{#1}^{(#2)}}
\newcommand{\similarity}[1]{\rho(#1) }
\newcommand{\cor}[2]{<\bar{#1}\mid\bar{#2}>}

\newcommand{\almostall}{\bar{\forall}}
\newcommand{\possibly}{\bar{\exists}}


\title{Rubbishについて}
\author{大村伸一}
\date{2020/10/19}				% Activate to display a given date or no date

\begin{document}
\maketitle

\section{概要}
このディレクトリのドキュメントは、Rubbishの実装や構想に関するものです。

そもそも自動証明がどんなことでも証明できるという気はしない。どのような証明ができるのか。知識をどのように表現すればよいのか。証明機構/推論機構の有益な使い方は何か。

「推論」という言葉がいろいろな文脈で登場し、「推論」という言葉の意味がだんだんわからなくなってきている。

\begin{description}

\item[代入.tex]
 代入はいろいろな実装方法があり、その表現もまたさまざまである。
 機能面と性能面で検討する。

\item[Unificationについて.tex]
 Unificationのアルゴリズムは同じでも、実装は代入の実装に依存する。

\item[rsolutionの方法.tex]
 Unificationにもとづくresolutionの実装では、clauseとliteralの性質に基づいたデータ構造を用いた。

\item[DVCの実現方法.tex]
 Disjoint Variable Condition(DVC)の実現方法について

\item[証明構造分析ツール.tex]
 目的はひとつの証明を作ることでなく、すべての証明の構造を用いて必要な証明をとらえること。

\item[機械学習と証明.tex]
 機械学習あるいは深層学習は、対象世界から述語を取り出す方法であり、それらの述語と証明から何か有益な結果が得られるとよい。そのような観点からツールを考える。
 真相学習というものはありうるか?

\item[観測事実の測定.tex]
 時が立つにつれて変化する命題がある場合、証明機構はどのように使えるのか。
 その命題が世界の同じ何かについての命題でなければ意味がない。同一性は機械学習がたとえば確率的にとらえるのではないか。

\end{description}

\end{document}
