\documentclass[10pt, oneside]{jarticle}   	% use "amsart" instead of "article" for AMSLaTeX format
\usepackage{geometry}                		% See geometry.pdf to learn the layout options. There are lots.
\geometry{a4paper}                   		% ... or a4paper or a5paper or ... 
%\geometry{landscape}                		% Activate for for rotated page geometry
%\usepackage[parfill]{parskip}    		% Activate to begin paragraphs with an empty line rather than an indent
%\usepackage{graphicx}				% Use pdf, png, jpg, or eps§ with pdflatex; use eps in DVI mode
								% TeX will automatically convert eps --> pdf in pdflatex		

\usepackage[all]{xy}
\usepackage{enumerate}
\usepackage{version}
\usepackage{amssymb}
\usepackage{amsmath}
\usepackage{cases}

\usepackage[dvipdfmx]{graphicx}


\usepackage{amssymb}
\usepackage{amsthm}

\theoremstyle{definition}
\newtheorem{theorem}{定理}
\newtheorem{proposition}{命題}
\newtheorem{hypothesis}{予想}
\newtheorem{problem}{課題}
\newtheorem*{theorem*}{定理}
\newtheorem*{proposition*}{命題}
\newtheorem*{hypothesis*}{予想}
\newtheorem{definition}[theorem]{定義}
\renewcommand\proofname{\bf 証明}


\newcommand{\ivec}[2]{\bar{#1}^{(#2)}}
\newcommand{\similarity}[1]{\rho(#1) }
\newcommand{\cor}[2]{<\bar{#1}\mid\bar{#2}>}

\newcommand{\almostall}{\bar{\forall}}
\newcommand{\possibly}{\bar{\exists}}


\title{有限世界の否定}
\author{omura}
%\date{}							% Activate to display a given date or no date

\begin{document}
\maketitle

\section{概要}
懐中電灯を書いていてると、有限ドメイン上の命題の否定が必要なきがしてきた。
たとえば、懐中電灯の述語を$FL(switch, lamp)$としたとき、ただし、$switchはスイッチの状態で\{on,off\}、lampは電灯の状態で\{dark, bright\}$とすると、正常な懐中電灯の動作は
\begin{equation}
FL(on, bright)
\end{equation}
\begin{equation}
FL(off, dark)
\end{equation}
になる。ここで、実際の懐中電灯の状態を観察してこういうFactが得られたとする。
\begin{equation}
FL(on, dark)
\end{equation}

このとき、正常な動作とFactが矛盾するところから、原因を調べるという話を作りたいとすると、(1)と(3)で矛盾してほしいが、これは異なる命題でありresolutionでは矛盾できない。

\subsection{命題の場合}
命題レベルだと、$P$と$Q$は記号が違えば否定されても関係がないのでresolutionでは矛盾できない

\subsection{一階述語の定数の場合}
まず、$P(a)$と$P(b)$を考える。aもbも定数とする。

このときも命題論理と同じく、同じ述語記号ではあるが否定にはならないので矛盾しない。
しかし、述語引数のドメインを考えると、つまりエルブラン宇宙は$\{a,b\}$になるので、$\neg P(a)$は$P(b)$と等しいのではないか。
だとすれば、$P(a) : P(b)$は$P(a) : \neg P(a)$であり、矛盾が導けると思う。

構文論でおさまるべきところ意味論まで踏み込んでいるので、数学的にはでたらめだが、アイディアということでそこは目をつぶる。

この矛盾には前提がいろいろあるので、一般的には成り立たないだろう。
\begin{description}
\item[ $\neg P(a)$] これはP(b)と関係がないので、P(b)だからといって$\neg P(a)$は成り立たない
\item[ ドメインと否定] $\{a,b\}$の$a$についての主張$P(-)$が否定されてドメインの差$D-\{a\}$ になるには、ドメインがsingletonであるとかいう条件が必要。
\item[ $\{P(a), P(b)\}$] では、$P(a)$も$P(b)$もaxiomに存在して、つまりsingletonでないときの否定はどうなるのか? ドメインが$\{a,b,c\}$なら$P(c)$になるのか? 
\item[ $\{P(a), P(b)\}$] ドメインが$\{a,b\}$ならそもそも否定はありえないのか?

おもしろいと思ったのだが、でたらめだった

\end{description}


\end{document}
